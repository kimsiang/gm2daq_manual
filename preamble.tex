\documentclass[12pt,letterpaper,oneside]{book}
\usepackage[english]{babel}
\usepackage[latin1]{inputenc}
\usepackage{graphicx}

\usepackage[margin=2.54cm,letterpaper]{geometry}
\usepackage[explicit]{titlesec}
\usepackage{titletoc}
\usepackage{tikz}
\usepackage{epigraph}
\usepackage{xpatch}
\usepackage{xspace}
\usepackage{lmodern}

\usepackage{array}
\usepackage{color}
\usepackage{parskip}
\usepackage{multirow}
\usepackage{amsmath}

\usepackage{courier} % for the code parts
\usepackage{hyperref} % for linking to URLs


% ----------------------------------------------------------------

% phrase styling
\def\gm2{{$g$$-$2}}
\newcommand{\uTCA}{$\mu$TCA\xspace}

\def\labelitemi{--} % use dashes for list items


\newlength\ChapWd
\settowidth\ChapWd{\huge\chaptertitlename}

% cornell red color
\definecolor{cornellred}{RGB}{179,27,27}
\definecolor{kentuckyblue}{RGB}{00,45,100}


\titleclass{\part}{top} % make part like a chapter
\titleformat{\part}[display]
{\normalfont\filcenter\sffamily}
  {\tikz[remember picture,overlay]{}
  }{0pt}{\fontsize{33}{40}\selectfont\color{cornellred}#1}[\vskip0pt\Large***]
\titlespacing*{\part}
  {0pt}{50pt}{10pt}


\titleformat{\chapter}[display]
  {\normalfont\filcenter\sffamily}
  {\tikz[remember picture,overlay]
    {
%    \node[fill=white,font=\fontsize{60}{72}\selectfont\color{cornellred},anchor=north east,minimum size=\ChapWd] 
%      at ([xshift=-15pt,yshift=-15pt]current page.north east) 
%      (numb) {\thechapter};
%    \node[rotate=90,anchor=south,inner sep=0pt,font=\huge] at (numb.west) {};%\chaptertitlename};
    }
  }{0pt}{\fontsize{33}{40}\selectfont\color{cornellred}#1}[\vskip0pt\Large***]
\titlespacing*{\chapter}
  {0pt}{50pt}{10pt}


\makeatletter
\xpatchcmd{\ttl@printlist}{\endgroup}{{\noindent\color{cornellred}\rule{\textwidth}{1.5pt}}\vskip30pt\endgroup}{}{}
\makeatother


\newcommand\DoPToC{
\startcontents[parts]
\printcontents[parts]{}{0}{\noindent{\color{cornellred}\rule{\textwidth}{1.5pt}}\par}
}


\setlength\epigraphrule{0pt}
\renewcommand\textflush{flushright}
\renewcommand\epigraphsize{\normalsize\itshape}


% ----------------------------------------------------------------
% for code format

\usepackage{listings}
\lstset{
language=bash,                   % choose the language of the code
basicstyle=\footnotesize\ttfamily,       % the size of the fonts that are used for the code
							% the \ttfamily causes the output to be in the courier font
numbers=left,                   % where to put the line-numbers
numberstyle=\footnotesize,      % the size of the fonts that are used for the line-numbers
stepnumber=1,                   % the step between two line-numbers. If it is 1 each line will be numbered
numbersep=5pt,                  % how far the line-numbers are from the code
backgroundcolor=\color{white},  % choose the background color. You must add \usepackage{color}
showspaces=false,               % show spaces adding particular underscores
showstringspaces=false,         % underline spaces within strings
showtabs=false,                 % show tabs within strings adding particular underscores
frame=single,                   % adds a frame around the code
tabsize=2,                      % sets default tabsize to 2 spaces
captionpos=b,                   % sets the caption-position to bottom
breaklines=true,                % sets automatic line breaking
breakatwhitespace=false,        % sets if automatic breaks should only happen at whitespace
escapeinside={\%*}{*)},         % if you want to add a comment within your code
literate={~} {$\sim$}{1},	% to make the tilde look right
deletekeywords={and, or, for, explicit, this, using, if, do, not, while, cd, export, source, env}	% to get rid of the bolding of keywords in C++
}


% ----------------------------------------------------------------
% for acronyms

\usepackage[intoc]{nomencl}
\makenomenclature
\renewcommand{\nomname}{Acronyms}
\setlength{\nomlabelwidth}{0.2\textwidth}


% ----------------------------------------------------------------
% for file tree structure

\makeatletter
\newcount\dirtree@lvl
\newcount\dirtree@plvl
\newcount\dirtree@clvl
\def\dirtree@growth{%
  \ifnum\tikznumberofcurrentchild=1\relax
  \global\advance\dirtree@plvl by 1
  \expandafter\xdef\csname dirtree@p@\the\dirtree@plvl\endcsname{\the\dirtree@lvl}
  \fi
  \global\advance\dirtree@lvl by 1\relax
  \dirtree@clvl=\dirtree@lvl
  \advance\dirtree@clvl by -\csname dirtree@p@\the\dirtree@plvl\endcsname
  \pgf@xa=1cm\relax
  \pgf@ya=-0.75cm\relax
  \pgf@ya=\dirtree@clvl\pgf@ya
  \pgftransformshift{\pgfqpoint{\the\pgf@xa}{\the\pgf@ya}}%
  \ifnum\tikznumberofcurrentchild=\tikznumberofchildren
  \global\advance\dirtree@plvl by -1
  \fi
}

\tikzset{
  dirtree/.style={
    growth function=\dirtree@growth,
    every node/.style={anchor=north},
    every child node/.style={anchor=west},
    edge from parent path={(\tikzparentnode\tikzparentanchor) |- (\tikzchildnode\tikzchildanchor)}
  }
}
